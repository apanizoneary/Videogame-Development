%\ Tema : Funciones
%\ Dificultad : Media
%\ ID : 1

\documentclass[10pt,spanish,hyperref={pdfpagelabels=false}]{beamer}
\DeclareMathOperator{\RR}{\mathbb{R}}
\DeclareMathOperator{\too}{\longrightarrow }
\usepackage[spanish]{babel}
\selectlanguage{spanish}
\usepackage[utf8]{inputenc}
\usepackage{lmodern}
\usepackage{graphicx}
\renewcommand*\footnoterule{}
\let\thefootnote\relax

\beamertemplatenavigationsymbolsempty 

\author{Alejandro Panizo}
\title{Pregunta \LaTeX}

\begin{document}

\Large

\rightskip=0pt

\begin{frame}

Sea $f :\RR ^2 \to \RR$ diferenciable en $(a,b)\in \RR ^2$, entonces existe la derivada direccional de $f$ en $(a,b)$ para cualquier vector $\overline{v}\in \RR ^2$
y es m\'{a}xima

\medskip



\begin{enumerate}[1.] \rightskip=0pt
\item Si $\overline{v}=\nabla f(a,b)$.
\item Si $\overline{v}$ es paralelo al gradiente.
\item Si $\overline{v}$ es perpendicular al gradiente.
\item Ninguna de las anteriores es correcta.
\end{enumerate}


\end{frame}

\end{document}
