%\ Tema : Funciones
%\ Dificultad : Media
%\ ID : 1

\documentclass[10pt,spanish,hyperref={pdfpagelabels=false}]{beamer}
\DeclareMathOperator{\RR}{\mathbb{R}}
\DeclareMathOperator{\too}{\longrightarrow }
\usepackage[spanish]{babel}
\selectlanguage{spanish}
\usepackage[utf8]{inputenc}
\usepackage{lmodern}
\usepackage{graphicx}
\renewcommand*\footnoterule{}
\let\thefootnote\relax

\beamertemplatenavigationsymbolsempty 

\author{Alejandro Panizo}
\title{Pregunta \LaTeX}

\begin{document}

\Large

\rightskip=0pt

\begin{frame}

Al realizar el siguiente l\'{\i}mite por rectas
$$ \lim _{(x,y)\to (0,0)} \dfrac{ xy}{1+x\cos (y)},$$
 se tiene:

\begin{enumerate}[1.] \rightskip=0pt
\item No depende de $m$ y por tanto el l\'{\i}mite es cero.
\item No depende de $m$ y con este resultado, no podemos asegurar  que el l\'{\i}mite doble exista.
\item Depende de $m$ y por tanto  el l\'{\i}mite doble no existe.
\end{enumerate}

\end{frame}

\end{document}
