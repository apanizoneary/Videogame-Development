%\ Tema : Funciones
%\ Dificultad : Media
%\ ID : 1

\documentclass[10pt,spanish,hyperref={pdfpagelabels=false}]{beamer}
\DeclareMathOperator{\RR}{\mathbb{R}}
\DeclareMathOperator{\too}{\longrightarrow }
\usepackage[spanish]{babel}
\selectlanguage{spanish}
\usepackage[utf8]{inputenc}
\usepackage{lmodern}
\usepackage{graphicx}
\renewcommand*\footnoterule{}
\let\thefootnote\relax

\beamertemplatenavigationsymbolsempty 

\author{Alejandro Panizo}
\title{Pregunta \LaTeX}

\begin{document}

\Large

\rightskip=0pt

\begin{frame}

Sea $f:\RR^n \to \RR $. El valor m\'{a}ximo de la derivada direccional:

\medskip

\begin{enumerate}[1.] \rightskip=0pt
\item  Se alcanza siempre en la direcci\'{o}n del gradiente.
\item Se alcanza siempre en la direcci\'{o}n opuesta a la del gradiente.
\item  Se alcanza siempre en la direcci\'{o}n perpendicular al gradiente.
\item   Ninguna de las  afirmaciones anteriores es co\-rrecta.
\end{enumerate}


\end{frame}

\end{document}
